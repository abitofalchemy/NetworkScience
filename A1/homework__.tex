\documentclass[letterpaper, 12pt, oneside]{memoir}

\usepackage{lipsum}
% Abstract: Remove indents around abstract text
\setlength{\absleftindent}{0pt}
\setlength{\absrightindent}{0pt}
% Change font size to conform with the rest of the document text
\renewcommand{\abstracttextfont}{\normalsize}
% Set left and right margins to equal ratio
\setlrmargins{*}{*}{1}

\abstractnum   % Format heading as chapter
\abstractintoc % Include "Abstract" in ToC

\title{Network Model Genesis Using Hyperedge Replacement Grammars}
\author{{Salvador Agui\~{n}aga}\\
        %\\
        %{Ph.D. Thesis Proposal}\\
        %\vspace{10pt}
        {University of Notre Dame}\\
        }


\begin{document}
\frontmatter

% Adapted from http://tex.stackexchange.com/a/187904/3345
\let\oldcleartorecto\cleartorecto
\let\cleartorecto\relax
\maketitle
\pagestyle{empty}
\aliaspagestyle{chapter}{empty}

%% 
%% Thesis abstract
%% 
\begin{abstract}
  %------%---------%---------%---------%---------%---------%---------%---------%
  %The Context
  %The Problem
  With the volume of published science and on-line supplementary information growing
  at a breakneck speed, more effective tools that go beyond search are 
  desperately needed. Scientists across many fields need to learn from, make sense
  of, and keep up with vast volumes of published literature.
  This ongoing challenge prompts us to continuously ask ourselves whether
  or not it is possible to build systems that enable scientists to ask
  and probe the big questions in science. To put it another way,
  is it suddenly possible to probe our codified body of knowledge
  to help transform and expand the way scientists think about the big challenges
  it their respective fields?
  Advancements in technology has made it possible to arrive at
  the crossroads where computation and computability meets what is desperately
  necessary in order to forge the foundation to great new breakthroughs in science.\\
  

  My research work focuses on deeper examination of how we codify and store
  knowledge. Decomposing massive networks of published science (JSTOR, ArXiv, etc.)
  to probe the underlying structure will enable us to build new computational tools.
  These tools will help the scientific community generate new hypotheses, reconsider
  ones previously shelved, or transform the way we think about the problems
  in light of new information.\\

  My contribution centers on the development of algorithms for network models. I
  am working on hyperedge replacement grammars to grow graphs (or networks) to
  any order and size. The graph generating framework learns the structure of real-world networks to produce a set of production rules. These rules are the foundation that
  allows us to grow new graphs by probabilistically selecting how we replace
  hyperedges until graphs are formed to a desired edge or node count. Properties
  observed in the real-world networks are preserved in the resulting graphs.
  

\end{abstract}
%  \let\cleartorecto\oldcleartorecto
%- The technical problem to be solved with a justification of its importance
%- An account of related and prior works that explains why these works have not solved the problem
%- The specific research problem or question that your thesis work will address
%- A sketch of the proposed approach or solution
%- The expected contributions of your dissertation research
%- Progress in solving the stated problem
%- The methods you are using or will use to carry out your research
%- A plan for evaluating your work and presenting credible evidence of your results to the research community
\section*{Thesis Summary}
%- The technical problem to be solved with a justification of its importance


%  \setcounter{tocdepth}{3}
%  \tableofcontents

%  \mainmatter

%  \chapter{One}
%  \section{Overview\label{sec:Overview}}
%  \lipsum[4]
\end{document}
