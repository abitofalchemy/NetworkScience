\documentclass[letterpaper,10pt]{article}

\usepackage{fancyhdr}
\usepackage{extramarks}
\usepackage{amsmath,mathtools}
\usepackage{amsthm}
\usepackage{amsfonts}
\usepackage{color}

% \usepackage{tikz}
% \usepackage[plain]{algorithm}
% \usepackage{algpseudocode}

% \usetikzlibrary{automata,positioning}

%
% Basic Document Settings
%

\topmargin=-0.45in
\evensidemargin=0in
\oddsidemargin=0in
\textwidth=6.5in
\textheight=9.0in
\headsep=0.25in

\linespread{1.1}

\pagestyle{fancy}
\lhead{\hmwkAuthorName}
\chead{\hmwkClass\ (\hmwkClassInstructor\ \hmwkClassTime): \hmwkTitle}
\rhead{\firstxmark}
\lfoot{\lastxmark}
\cfoot{\thepage}

\renewcommand\headrulewidth{0.4pt}
\renewcommand\footrulewidth{0.4pt}

\setlength\parindent{0pt}

%
% Create Problem Sections
%

\newcommand{\enterProblemHeader}[1]{
	    \nobreak\extramarks{}{Problem \arabic{#1} continued on next page\ldots}\nobreak{}
		    \nobreak\extramarks{Problem \arabic{#1} (continued)}{Problem \arabic{#1} continued on next page\ldots}\nobreak{}
}

\newcommand{\exitProblemHeader}[1]{
	    \nobreak\extramarks{Problem \arabic{#1} (continued)}{Problem \arabic{#1} continued on next page\ldots}\nobreak{}
		    \stepcounter{#1}
			    \nobreak\extramarks{Problem \arabic{#1}}{}\nobreak{}
}

\setcounter{secnumdepth}{0}
\newcounter{partCounter}
\newcounter{homeworkProblemCounter}
\setcounter{homeworkProblemCounter}{1}
\nobreak\extramarks{Problem \arabic{homeworkProblemCounter}}{}\nobreak{}

%
% Homework Problem Environment
%
% This environment takes an optional argument. When given, it will adjust the
% problem counter. This is useful for when the problems given for your
% assignment aren't sequential. See the last 3 problems of this template for an
% example.
%
\newenvironment{homeworkProblem}[1][-1]{
	    \ifnum#1>0
		        \setcounter{homeworkProblemCounter}{#1}
				    \fi
					    \section{Problem \arabic{homeworkProblemCounter}}
						    \setcounter{partCounter}{1}
							    \enterProblemHeader{homeworkProblemCounter}
}{
	    \exitProblemHeader{homeworkProblemCounter}
}

%
% Homework Details
%   - Title
%   - Due date
%   - Class
%   - Section/Time
%   - Instructor
%   - Author
%

\newcommand{\hmwkTitle}{Homework\ \#1}
\newcommand{\hmwkDueDate}{Feb 08 2016}
\newcommand{\hmwkClass}{Intro to Network Science}
\newcommand{\hmwkClassTime}{}
\newcommand{\hmwkClassInstructor}{Z. Toroczkai}
\newcommand{\hmwkAuthorName}{Sal Agui\~{n}aga}
\newcommand{\hmwkNetID}{NetID: saguinag}

%
% Title Page
%

\title{
	\vspace{2in}
	\textmd{\textbf{\hmwkClass:\ \hmwkTitle}}\\
	\normalsize\vspace{0.1in}\small{Due\ on\ \hmwkDueDate\ by 5:00pm}\\
	\vspace{0.1in}\large{\textit{\hmwkClass\ $\cdot$ \hmwkClassInstructor\ $\cdot$ Spring 2016}}
	\vspace{3in}
}

\author{\textbf{\hmwkAuthorName}\\
\textbf{\hmwkNetID}}

\renewcommand{\part}[1]{\textbf{\large Part \Alph{partCounter}}\stepcounter{partCounter}\\}

%
% Various Helper Commands
%

% Useful for algorithms
\newcommand{\alg}[1]{\textsc{\bfseries \footnotesize #1}}

% For derivatives
\newcommand{\deriv}[1]{\frac{\mathrm{d}}{\mathrm{d}x} (#1)}

% For partial derivatives
\newcommand{\pderiv}[2]{\frac{\partial}{\partial #1} (#2)}

% Integral dx
\newcommand{\dx}{\mathrm{d}x}

% Alias for the Solution section header
\newcommand{\solution}{\textbf{\large Solution}}

% Probability commands: Expectation, Variance, Covariance, Bias
\newcommand{\E}{\mathrm{E}}
\newcommand{\Var}{\mathrm{Var}}
\newcommand{\Cov}{\mathrm{Cov}}
\newcommand{\Bias}{\mathrm{Bias}}

\begin{document}

\maketitle
\pagebreak
%% 1.
\begin{homeworkProblem}
%% Rerences:
%% https://books.google.com/books?id=K6-FvXRlKsQC&pg=PA19&lpg=PA19&dq=A+graph+G+of+order+26+and+size+58&source=bl&ots=Xz9eigj6cH&sig=7OvbcjDr8N8msZd60C-GFj3N4_k&hl=en&sa=X&ved=0ahUKEwjh_MXX4tnKAhWkg4MKHZpkD_EQ6AEIJDAB#v=onepage&q=A%20graph%20G%20of%20order%2026%20and%20size%2058&f=false
%% http://www.math.kit.edu/iag6/lehre/graphtheo2013w/media/graphtheory.pdf
\section{Solution:}
From the so called Handshaking Lemma: In any graph, the sum
of all the vertex-degree is equal to twice the number of edges.
In this case n=26  and m=58, therefore given the initial vertex-degree
set (or sequence) 
$$\{5: deg(v) =4, 6: deg(v) = 5, 7: deg(v)=6\}$$
which adds up to 18 vertices, leaving us with 8 vertices, and from the First 
Theorem of Graph Theory we have that if G is a graph of orer n and size m, then
$$ \sum_{i=1}^n deg~v_i = 2m$$ 
means that the remining vertices (8 of them) will each have to be of $(2*m) - (5*4+6*5+7*6) = 24/8 = 3$.
Three vertices each of the 8 remaining vertices will be their vertex-degree. 
\end{homeworkProblem}
%% 2.
\begin{homeworkProblem}
\section{Solution}
By observation, graph $G_1$ and $G_2$ are of the same order and size, their respeive vertex
degree are the same and consisten with Theorem 2 (Lecture 3: Fundamentals of graph theory II.), 
thus these graphs are isomorphic.
  \begin{align*} 
n &=  7 \\ 
m &=  10
  \end{align*}
  $$G_1 \simeq G_2 $$

By the same process of observation and analysis,
  $$ H_1 \simeq H_2 $$
\end{homeworkProblem}

%% 3.
\begin{homeworkProblem}
Given graph G with order n = 3k + 3 for some positive integer k.
Every vertex of G has degree k + 1, k+2 and k+3. 
Prove that G has at least k+3 vertices of degree k+1 or 
at least k+1 vertices of degree k+2 or at least k + 2 vertices of degree k + 3.
% http://www.math.ku.edu/~huneke/HWanswers725-7.pdf
% https://www.cs.cmu.edu/~adamchik/21-127/lectures/graphs_1_print.pdf
% http://www.math.umn.edu/~akhmedov/M4707Finals.pdf

\section{Solution:}
%Given: G has order 3k+3 and that every vertex degree is k+1, k+2, k+3.
%To prove that G has at least k+3 of degree  k+1:
\noindent\begin{tabular}{@{}*{1}{p{0.9\textwidth}@{}}}
It is possible if and only if n and r (the degree of each vertex)  are not both odd integers.
\\ 
Proof that G has at least k+3 vertices of degree k+1 \\\hline 

Given the initial conditions for graph G, we are dealing with r-regular graphs
where if we restrict r to even integers, then to show that G has at least 
k+3 vertices of degree k+1, let can say that:\\
Let $k = 2$, then $n = 3k + 3 =  9$,
$k+3$ vertices totaling 5 may each have degree 3, $5*3=15$. \\
From Corollary 1.5 [Charttrand et al.] $0< \delta(3) \le n-1 = 8$, this requirement
is maintained and we can have the remainder of the nodes with the same degree will
still add up to a total number of edges to be within $n-1$.\\
\\
Proof that G has at least k+1 vertices of degree k+2.\\\hline
A graph G of order n, where n = 3k+3, if we let n be an odd integer
and again let k = 2, then the number of vertices is an odd integer.\\
Out of $n =  3k+3 = 3(2)+3 = 9$ vertices, 
$(k+1) = 2+1 = 3 $ nodes of degree $k+2 = 2+2 = 4$ results in a total of 12 edges.\\
With the remaining 6 vertices, each of degree 4 we end up with a total of 36 connected
stubs, which results in 18 edges well below the n(n-1)/2 max.\\ 
\\
Proof: \\\hline
Let k = 0, with n = 3k+3, n = 3\\
For (k=0)+2 = 2 vertices each of degree (k=0) + 3 = 3 (i.e., r)\\
NB: Notice both n and r are both odd.\\
Leaves us with 1 node of degree 3, but since n=3, the following would be
violated: $0 < \delta(v)=3 \le (n-1) = 2$.\\
On the other hand, if we let k = 1, we end with n = 6, which is even.\\
Where k+2 (or 3) vertices each of degree k+3 = 4 results in deg(v) being
even, in this case Theorem 1.7~[1] holds.
\end{tabular}
\end{homeworkProblem}
%%
%% 4.
\begin{homeworkProblem}
% http://www.doc88.com/p-9055292645283.html
Show that any graph contains a path of length $\delta(G)$ and a cycle of length at least           
$\delta(G) + 1$, if $\delta(G) \ge 2$.
\section{Solution:}
First, the minimum degree of G is denoted by $\delta(G)$.

\noindent\begin{tabular}{@{}*{2}{p{0.5\textwidth}@{}}}
%For two nodes with edge between them if we add another node and connect the them and labe
%the nodes $v_1$, $v_2$, $v_3$, then we  &\\
Pick an arbritrary vertex and label it $v_1$. &\\
Next, select a neighbor and label it $v_2$. &\\
Then select any of its other neighbors (different
from $v_1$) and label it $v_3$.  Now we have a path of length $\delta(G)$ corresponding to 
length of 1. If we continue like this, we can always select a path of length $\delta(G)$ & \\
To show that this graph can have a cyle of length at least $\delta(G)$ + 1, which 
in this case it would be a path of length 2 and abide with the rule that $\delta(G) \ge 2$,then
we connect $v_3$ to $v_1$ results in a graph with a path of maximal length. Now,
each node is of degree 2, which is also the minimal node degree and the cycle is of 
length $\delta(G)+1$. Note that if we do not connect it back to $v_1$ the graph does not 
result in graph with $\delta(G) = 2$ in this case.&\\
\end{tabular}
\end{homeworkProblem}

%% 5.
\begin{homeworkProblem}
Show that for every finite set S of positive integers, there exists a positive 
integer k such that the sequence obtained by listing each element of S k times 
is graphical. Find the smallest such k for $S = \{2, 6, 7\}$.
\section{Solution:}
To show that any degree sequence is graphical, the following conditions will be
met: 
\begin{enumerate}
\item The degree of any vertex $v_i$ shall be less than or equal to n-1, for $i (1\le i \le n)$.
\item The Sum of the sum of all deg $v_i$, for i 1 through n, shall be even.
\item G must also have even number of of vertices (thus in the example below there has of be 
a k of deg v = 7 which has to be even.)
\item In addition, the sequence must hold up to the Havel-Hakimi Theorem [Chartrand et al.]. To 
illustrate the theorem using the sequence obtained by listing each element of S k times:
if k = 2 
  \begin{minipage}[c]{0.8\textwidth}
  $$ s: 2,2,6,6,7,7$$
  Reordering the sequence:
  $$ s: 7,7,6,6,2,2$$
  After one application of the theorem, e.g., deleting 7 and subtracting 1 from the next 7 items, 
  which we lack, then we change k to 4 and try again.
  $$ s: 2,2,2,2,6,6,6,6,7,7,7,7$$
  Reordering the sequence:
  $$ s: 7,7,7,7,6,6,6,6,2,2,2,2$$
  Applying the Havel-Hakimi Theorem:
  \begin{align*}
  s_1':& 6,6,6,5,5,5,5,2,2,2,2\\
  s_2':& 5,5,4,4,4,4,2,2,2,2\\
  s_3':& 4,3,3,3,3,2,2,2,2\\
  s_4':& 2,2,2,2,2,2,2,2\\
  s_5':& 1,1,2,2,2,2,2\\
  s_6':& 0,1,2,2,2,2\\
  s_7':& 1,2,2,2,2,0\\
  s_8':& 1,2,2,2,0\\
  s_9':& 1,2,2,0\\
  s_A':& 1,2,0\\
  s_B':& 1,0\\
  s_C':& 0\\
  \end{align*} 
  Therefor,  $s$ is graphical. the smallest $k = 4$. 

  \end{minipage}
  \end{enumerate} 
\end{homeworkProblem}


%% 6.
\begin{homeworkProblem}
\section{Solution:}
\end{homeworkProblem}

\section{References}
\noindent Chartrand, Gary and Lesniak, Linda and Zhang, Ping;
  \textit{Graphs \& Digraphs}, Fifth Edition; 
    2010; Chapman \& Hall/CRC
 
\end{document}
